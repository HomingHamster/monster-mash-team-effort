\documentclass{article}

\usepackage[hmargin=2cm,vmargin=2.5cm]{geometry}

\begin{document}
\section{Overview of Proposed System}

\subsection{Design Choices}
We were tasked to make a website, there are a few decisions that had to be made relating to the design of the website that are listed here.

\subsubsection{Using the web as a platform}
Using the web as a platform for this app means that the pages will be accessable by a large number of people, this means that the potential audience is larger because it is not limited to one machine like a Windows PC. The data should be centralised for this game to work, and this is ideal because it can be stored on a central machine or server.

The web is fairly easy to develop for, there are a number of frameworks and platforms out there to use, this means that development can be faster and more instantly accessable by users. Designing websites for the web can be tricky and requires a good design sense and some exerience. We should choose a method that allows us to keep the same design consistant throughout the site and that allows us to keep it up to date easily; web design trends change quickly.

\subsubsection{High Level}
We chose to use tools that were high level, at a small performance cost. There are a number of reasons for doing this. Firstly because it allows developers to develop at a higher speed, which means less time until there is a working solution. It means less time working out bugs in the code because higher level tools can be more intuative and often easier to debug. Using high level tools can be good for security if the platform has been tested thourhgly because it takes the burden away from the developer; most of the issues have already been dealt with.

As well as being easy to learn for the developers at this stage it is also useful for anyone who has to maintain the code afterward, less time is needed to figure out what something does and it's therefore easier to fix. The only real downside is that there is a performance sacrifice at some stages, it was judged on all tools that we used that the performance cost is worth the time spent less on development.

We chose to use the bootstrap library for creating a consistant design we also used jquery to help make functional javascript more quickly. In java we are going to use DERBY server and client in order to reduce the amount of SQL that we have to write to use the database.

\subsubsection{User centric design}
The application will work by presenting the user with a number of options when the user selects an option a new option will be presented. The user is always at the center of what is going on so we are going to make sure that they are presented all of the actions they would expect and that what the user is expecting to do should come first.

\subsubsection{Accessable Aesthetics}
The ui needs to be clear, the links will all be obviously clickable and everything needs to be where the user would expect it to be. The UI needs to work on many platforms, we anticipate 20\% of our traffic will be from mobile devices so we need to make sure we have a responsive design to suit it. As well as mobile browsers we need to cosider the difference between a lot of modern browsers and we need to make sure that we support them all and that the design is consistiant between all of the modern browsers. We are using the bootstrap library to help with some of these issues. 

\subsection{Choice of web platform}
We wanted our platform to be well documented and have materials to help beginners to pick up the platform. We have decided to use the Glassfish Open Source version as our platform. It needed to be able to produce website such that multiple people could develop for it. We wanted to be able to deploy the website to a server somewhere and a platform that is well supported in production enviroments and choosing a platform to support this was important to us.
We did not consider the Oricle Glassfish Server because it would cost to use it for the project.

We considered the following:
\begin{itemize}
	\item{Glassfish Open Source Server}
	\item{Google App Engine}
	\item{Apache Tomcat}
\end{itemize}

\subsubsection{Glassfish Open Source Server}
There are a number of benifits to this software above the other options.
The main two reasons for using this server are because it is open source
and because some members of our group have previous experience with it.
Another reason is because we expect there will be support for this
enviroment available. Both from the university and from the contributers
to the glassfish project. Glassfish has many more features than Tomcat,
the other open source option.

Spike testing was carried out and it was found that this peice of software
was easy to use and appropriate for the nature of our project.

\subsubsection{Google App Engine}
The main reason we didn't choose this software is because it prooved
unreliable in tests. This software is also closed source and using it
would mean that you rely upon google when the application is released.

\subsubsection{Apache Tomcat}
Tomcat was not as fully featured as Glassfish, and no one in the group 
has ever used it before, so there would be a steeper learning curve for
them and there would be no "in-group" support for using it.


\subsection{High Level Archetecture}

\subsubsection{Version Control}
For version control we are using Git. Git is a distributed version control
system, which some members of the group already have experience with.
Distributed version control systems give a slightly different development
pattern which suited the qualities of a group better than SVN.

Version control systems we considered:
\begin{itemize}
	\item{Git}
	\item{Bazaar}
	\item{Subversion}
\end{itemize}

\subsubsection{Integrated Development Enviroment}
We have decided to use the NetBeans IDE, because it is available free
and it is the preference of the majority of the group. Modules are
available for NetBeans to help with Version Control (Git) and JUnit.

IDEs considered:
\begin{itemize}
	\item{Eclipse}
	\item{NetBeans}
\end{itemize}

\subsubsection{Documentation Tool}
We decided to use \LaTeX{} because it is widely supported, there is a
template provided, and because it was prefered by the majority of the.

Methods of documentation we considered:
\begin{itemize}
	\item{\LaTeX{}}
	\item{Open Office/Libre Office}
	\item{Microsoft Word}
\end{itemize}


\subsection{Description of Target User}
The target user will be young people. Typically aged between 11 and 16.
We will have to make sure that no complicated language is used without good
reason and we will have to make sure that all content is appropriate. Other
things to consider, are:
\begin{itemize}
	\item{Make sure that there are no really lengty tasks to do}
	\item{Make sure that it will fit around the lifestyle of a young
		person of that age. ie. Around school, limited access
		to a computer.}
\end{itemize}
\end{document}
