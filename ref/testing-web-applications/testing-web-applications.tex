\documentclass{beamer}

\ifx\themename\undefined
  \def\themename{Berlin}
\fi

\usetheme{\themename}

\beamertemplatetransparentcovered

\usepackage{times}
\usepackage[T1]{fontenc}

\usepackage[normalem]{ulem}

\title{Testing Web Applications.}
\subtitle{How should we test the Monster Mash game?}

\author{Felix Farquharson}
\institute{Group Project\\ CS2SOMETHING}

\begin{document}

\begin{frame}
  \titlepage
\end{frame}

\begin{frame}
  \frametitle{Outline}
  \tableofcontents
\end{frame}

\section{Different types of testing.}
\subsection{Overview}

\begin{frame}<1>
  \frametitle{The different types of testing.}

  \begin{block}{I've done some research...}
    ... Well, I googled it at least! I was thinking about a 
	way of testing everything that was reliable and 
	worked for all the different parts of the application
	all in one.
  \end{block}

\

As it turns out we should probably test each part of the program
in a way that suits that specific part.

\end{frame}

\subsection{List of types}

\begin{frame}<2>
  \frametitle{A list of different types}
  \framesubtitle{Courtesy of http://www.softwaretestinghelp.com/}

  \begin{enumerate}
	\item Functionality Testing
	\item Usability testing
	\item Interface testing         
	\item Compatibility testing     << This is probably irrelivant.
	\item Performance testing	<< These are probably
	\item \sout{Security testing} 	<< both just for wow.
  \end{enumerate}
\end{frame}


\section{Suggested Tests}
\subsection{Functionality Testing}

\begin{frame}<3>
\frametitle{Testing the Functionality}
\begin{itemize}
\item Now, I took this to mean testing the backend.
\item Testing the code that makes the application do somthing useful.
\item This is where JUnit might \emph{acctually} be of some use. We might use it
in a fashion like Richard Shipman described in his lecture today 
(15/10).\\
Ie. write the tests and then develop to make sure the 
tests are forfilled. (Maybe with some overlap to allow development
to begin immediately.)
\item Other useful tests might be Test Tables and manual testing.
\end{itemize}
\end{frame}

\subsection{Useablity Testing}

\begin{frame}<4>
\frametitle{Testing the Useability.}

\begin{itemize}
\item I took this to mean largely testing the UI.
\item making sure all the links are intuative.
\item Making sure the application isn't too complicated 
	for normal human beings.
\end{itemize}
\begin{block}{How?}
For this I think that testing with real people is going to be 
very important. Preferably people who have never seen the program
before.
\end{block}

Making sure the UI designers understand what other people not in
our group see will probably help.
\end{frame}

\subsection{Interface Testing}
\begin{frame}<5>
\frametitle{Testing the interface.}
\begin{itemize}
\item Largely the same as Useability testing (I struggled to see
any difference at all).
\item Google define = testing the human interaction of a product
\item we run command-line tools, automated web browsers, 
or something else automatic...
\item See http://seleniumhq.org/ (automates web browsers).
\item I thought testing in different web browsers might be 
important here. (see http://browsershots.org/)
\end{itemize}
\end{frame}

\subsection{Compatability Testing}
\begin{frame}<6>
\frametitle{Testing compatibilty.}
\begin{block}{My understanding}
While this type of testing is probably refering to checking it 
works on all browsers or the server runs in all places. Most of
this task is taken away because of the fact Java is multiplatform.\\
Also I have covered the browser differences in the previous slide.
\end{block}
\begin{itemize}
\item I propose we use this title as a way to test that the server 
will interact with the other groups servers.
\item I couldn't think of a way to test this until we have defined
somthing to test it by. ie. the method of communication between 
servers.
\end{itemize}
\end{frame}

\subsection{security and performance testing}
\begin{frame}<7>
\frametitle{Testing for security and for performance.}
While these are both important, we have been told that we should
not be concerned about security. I think it's still good practice,
so we could run a few simple automated tests. (See Nessus on 
http://www.tenable.com/).\\
Performance is not likely to be an issue, and we are limited with 
our platforms and resources etc. So we can leave this for later if
appropriate, I think.
\end{frame}

\section{Questions.}

\begin{frame}<8>
\frametitle{Questions?}

\begin{block}{}
Are there any questions?
\end{block}

\end{frame}

\end{document}
